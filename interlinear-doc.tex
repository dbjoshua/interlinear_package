\documentclass{article}
%\usepackage{interlinear}
\usepackage{xparse}
\usepackage{verbatim}

% Commande pour afficher du code (sans les délimiteurs `d```)
\NewDocumentCommand{\code}{m}{\texttt{#1}}
\def\interlinearstyle{$\backslash$interlinearstyle}

% Commande pour du texte en gras
\NewDocumentCommand{\bftext}{m}{\textbf{#1}}

% Commande pour liste avec itemize
\NewDocumentCommand{\itm}{m}{\begin{itemize}\item #1\end{itemize}}

% Metadata
\title{interlinear Package Documentation}
\author{Akpoué Kouamé Josué}
\date{\today}

\begin{document}

\maketitle

\section{Package Description}
\subsection{Purpose}
The \code{interlinear} package provides a new interface for typesetting interlinear texts. It is based on \code{cgloss4e.sty} and is designed to avoid conflicts with other packages, such as \code{linguex}. This package allows for the simultaneous use of \code{linguex} and \code{interlinear} by renaming commands from \code{cgloss4e.sty} for better clarity and maintenance. The package is part of a larger set of tools to facilitate linguistic document preparation using LaTeX.

\subsection{What's New}

\itm{\bftext{Toolbox Integration:} Support for SIL Linguists Field's Toolbox input (with some modifications). You can copy-paste the interlinear text from Toolbox directly into LaTeX.}
\itm{\bftext{Four-Line Option:} Adds support for a fourth line (Part of speech) in interlinear texts.}
\itm{\bftext{Markers:} New markers are provided for various parts of the example, including \code{\textbackslash ct} (context), \code{\textbackslash li} (language information), \code{\textbackslash lt} (literal translation), and \code{\textbackslash gj} (grammaticality judgments).}
\itm{\bftext{Environment:} Defines an \code{interlinear} environment with options for fine-grained control over element appearance.}

\section{Installation and Loading}
To install, place \code{interlinear.sty} in a directory where LaTeX can find it. Load the package using:
\begin{verbatim}
\usepackage{interlinear}
\end{verbatim}
Alternatively, use:
\begin{verbatim}
\input{path/to/interlinear.sty}
\end{verbatim}

%\subsection{Required Packages}

To ensure the proper functioning of the \texttt{interlinear} package, the following LaTeX packages are required:

\begin{itemize}
    \item \texttt{marginnote}: This package is used for creating margin notes in your document.
    \item \texttt{xifthen}: Provides conditional commands for handling various logic in the package.
    \item \texttt{xkeyval}: Used to implement key=value options for customizing the behavior of the \texttt{interlinear} environment.
    \item \texttt{xparse}: Facilitates the definition of commands and environments with a more flexible interface.
    \item \texttt{etoolbox}: Supplies advanced tools for programming within LaTeX, such as patching and extending existing commands.
    \item \texttt{enumitem}: Allows customization of list environments (like \texttt{itemize}, \texttt{enumerate}, etc.), which may be used within \texttt{interlinear}.
\end{itemize}

Ensure that these packages are included in your LaTeX distribution.

\section{Basic Usage}
The main feature of \code{interlinear} is the \code{interlinear} environment. This environment allows you to structure interlinear texts using markers. Below is a comparison between \code{cgloss4e} and \code{interlinear} syntax.

\subsection{Example 1: Two-Line Interlinear}
\textbf{cgloss4e} code:
\begin{verbatim}
\gll lorem ipsum dolor\\
     sit amet consegur\\
\glt `Lorem ipsum dolor'
\end{verbatim}

\textbf{interlinear} code:
\begin{verbatim}
\begin{interlinear}
\tx lorem ipsum dolor\\
\gl sit amet consegur\\
\ft `Lorem ipsum dolor'\\
\end{interlinear}
\end{verbatim}

\subsection{Example 2: Three-Line Interlinear}
\textbf{cgloss4e} code:
\begin{verbatim}
\glll lorem ipsum\\
      dolor sit\\
      amet consegur\\
\glt `Lorem ipsum dolor'
\end{verbatim}

\textbf{interlinear} code:
\begin{verbatim}
\begin{interlinear}
\tx lorem ipsum\\
\mb dolor sit\\
\gl amet consegur\\
\ft `Lorem ipsum dolor'\\
\end{interlinear}
\end{verbatim}

\subsection{Complete Example}
\begin{verbatim}
\begin{interlinear}
\li Language\\
\rf Identifier of the data in the corpus/corpora\\
\ct Context\\
\gj Grammaticality judgment\\
\tx Object language text\\
\mb Morpheme breaks\\
\gl Glosses\\
\ps Part-of-speech\\
\ft Free translation\\
\lt Literal translation\\
\nt Extra note\\
\end{interlinear}
\end{verbatim}

\section{Setting Options}
The \code{interlinear} environment can be customized using the options described below.

\subsection{Interlinear Lines Number}
The number of interlinear lines is controlled by the \code{linesnumber} option. Possible values are 0, 2, 3, 4 (default is 2).

\subsection{Markers and Customization}
You can define custom markers for each part of the interlinear text:
\begin{verbatim}
\li languageinfomarker
\rf referencemarker
\ct contextmarker
\gj gramjudgmarker
\tx textmarker
\mb morphbreakmarker
\gl glossmarker
\ps partofspeechmarker
\ft freetranslationmarker
\lt literaltranslationmarker
\nt extranotemarker
\end{verbatim}

\subsection{Appearance Customization}
You can modify the appearance of different elements using the following options:
\begin{verbatim}
lineoneformat    % First line format
linetwoformat    % Second line format
linethreeformat  % Third line format
linefourformat   % Fourth line format
exformat         % Example text format (no-interlinear mode)
extranoteformat  % Extra note format
contextnameformat % Context label format
\end{verbatim}

These options can control the font family, shape, series, size, and color. To specify multiple formatting options, enclose them in a group. For example, the command \code{lineoneformat=\{$\backslash$itshape$\backslash$bfseries$\backslash$small\}} will set the text on the first interlinear or gloss line to boldface italics in a small size. By default, all formats are set to \code{$\backslash$normalfont}.

\subsection{Other Settings}
Various other elements can be customized:

\itm{The label indicating the context of the example can be changed using the \code{contextname} option. For instance, to change \code{Context:} to \code{Situation:}, use \code{contextname={Situation:}}.}

\itm{The space between the context and the body of the example can be specified via the \code{contextvoffset} option.}

\itm{By default, marginal notes are enclosed in parentheses \code{()}. To change this behavior, use the \code{extranoteleftpunct} and \code{extranoterightpunct} options. For example, to change the parentheses to square brackets \code{[]}, set \code{extranoteleftpunct={[}} and \code{extranoterightpunct={]}}. If you prefer no punctuation marks around your notes, set \code{extranoteleftpunct={}} and \code{extranoterightpunct={}}.}

\itm{The placement of language information for the linguistic example is controlled by the \code{languageinfopos} option, which accepts two values: \code{head} and \code{margin}. The default value, \code{margin}, places this information in parentheses in the margin of the first line of the example. The \code{head} value places it in a separate line at the very beginning of the example, even before the context. When the no-interlinear mode is enabled, \code{languageinfopos} should be set to \code{head} to avoid positioning conflicts with marginal notes.}

\subsection{Applying Options Outside the Environment}
Options can also be specified outside the \code{interlinear} environment using the \code{\interlinearstyle} command, which takes one mandatory argument. This argument should contain the list of chosen options, and \code{\interlinearstyle} must be called before the environment.

For example:
\begin{verbatim}
\begin{interlinear}[linesnumber=3,linefourformat=\itshape]
(content)
\end{interlinear}
\end{verbatim}

is equivalent to:
\begin{verbatim}
\interlinearstyle{linesnumber=3,linefourformat=\itshape} 
\begin{interlinear}
(content)
\end{interlinear}
\end{verbatim}

The \code{$\backslash$interlinearstyle} command redefines the default values for the specified options. Therefore, each time this command is used, it updates the default values for the specified options.

\subsection{Using Predefined and Custom Styles}
One of the features of the \code{interlinear} package is the ability to define multiple styles and use them throughout the document without having to call \code{$\backslash$interlinearstyle} each time. There are four predefined styles:

\begin{itemize}
    \item \code{default}: Standard settings.
    \item \code{gsr}: Sets the first interlinear line in italics.
    \item \code{nointerlinear}: Configures the no-interlinear mode, changes the \code{$\backslash$ft} marker to \code{$\backslash$ot}, and sets \code{languageinfopos} to \code{head}. This style provides a safe way to configure the no-interlinear mode without worrying about the various parameters.
    \item \code{gsrnointerlinear}: Similar to \code{nointerlinear}, but the text of the example is italicized.
\end{itemize}

There are two ways to apply a style:
\begin{enumerate}
\item Using the \code{$\backslash$UseInterlinearStyle\{\}} command before the \code{interlinear} environment, at the beginning of the document, or in the preamble.
\item Using the dedicated \code{style} option in the optional argument of the \code{interlinear} environment.
\end{enumerate}

Example: \code{$\backslash$begin\{interlinear\}[style=gsr]}.

It is recommended to specify the \code{style} option first when combining it with other options.

You can also specify additional options alongside the style to customize it as needed (e.g., if you want to temporarily modify a style parameter).

For example, the \code{gsr} style was declared using the following code:
\begin{verbatim}
\DeclareInterlinearStyle{gsr}{%
  linesnumber=2,%
  lineoneformat=\itshape,%
  linetwoformat=\normalfont%
}
\end{verbatim}

\end{document}
